\chapter{Requisitos}

Os requisitos de um projeto são as descrições do que o sistema deve fazer, os
serviços oferecidos e as restrições a seu funcionamento \cite{sommerville}.
Além de descrever as necessidades a serem cumpridas, os requisitos também são responsáveis
por determinar a qualidade que deve ser apresentada \cite{robertson}.
Baseado nisto, este capítulo tem por objetivo listar os requisitos presentes
no projeto.

\section{Requisitos Gerais}

\begin{enumerate}
\item Desenvolver a estrutura da cadeira.
\item Atender portadores de mobilidade reduzida, especificamente os paraplégicos.
\item O sistema precisa estar conectado à internet.
\end{enumerate}

\section{Subsistema - Controle e Monitoramento}

\begin{enumerate}[resume*]
\item O controle de movimentação da cadeira se dará por meio de um \textit{joystick}.
\item O sistema fará o monitoramento dos seguintes sinais vitais:
\begin{enumerate}[resume*]
\item Sinal 1
\item Sinal 2
\item Sinal 3
\end{enumerate}
\item Possibilidade de notificar algum parente.
\item Interação com recursos via aplicativo.
\item O sistema deve ser capaz de notificar algum
responsável quando um dos módulos essenciais para o funcionamento não estiver
funcionando corretamente
\end{enumerate}

\section{Alimentação}

\section{Estrutura}

\begin{enumerate}[resume*]
\item Estabelecer capacidade máxima de carga.
\item A estrutura deve atender aos princípios de ergonomia presente na NBR 9050\footnote{\url{http://www.ufpb.br/cia/contents/manuais/abnt-nbr9050-edicao-2015.pdf}}.
\end{enumerate}
