\begin{itemize}
    \item \textbf{Lucas Castro}
    \begin{quote}
        Sou integrante do subsistema de Estruturas, aluno de engenharia Automotiva. Como integrante do subsistema, fiquei responsável pela interpretação das simulações modais e estáticas feitas no software SolidWorks com o objetivo de assegurar que a estrutura da cadeira iria aguentar de fato todos os esforços estáticos (peso do usuário, bateria, sensores, equipamentos eletrônicos e gaveta) e se a vibração do motor acoplado à cadeira não iria comprometer a estrutura com a sua vibração. Além disso, em conjunto com os outros integrantes montamos todas as modificações que deveriam ser integradas à estrutura da cadeira para fazer com que ela funcione. O sistema motor-roda de atrito-roda da cadeira, sistema de gaveta para comportar bateria e equipamentos eletrônicos, berço de sensores e apoio de mão para o suporte do joystick foram construídos todos em conjunto, com o auxílio do professor Rhander e os técnicos do Galpão da FGA. Foi realizada a manufatura dos estofados utilizados para suporte e encosto da cadeira de roda, com o apoio da Brasilia Flex foi feita a espumação, corte da espuma e costura dos encostos. 
    \end{quote}
\end{itemize}
