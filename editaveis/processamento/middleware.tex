\section{Middleware}

O \textit{middleware} do subsistema, uma Raspberry, tinha como resultados
esperados uma aplicação que pudesse, ao rodar no embarcado, receber sinais e
enviá-los de maneira correta ao servidor.

Os resultados esperados foram atingidos. Foi desenvolvido a aplicação
Shoelace\footnote{\url{https://github.com/cadeiracuidadora/shoelace}}, que
serve como abstração para a aquisição dos dados do conversor A/D e que envia
os resultados para um servidor remoto.

Definimos que uma regra de negócio deveria ser que toda Raspberry tivesse
um \textit{token} e uma senha incluída, e esses dados são então utilizados nas
requisições para os servidores. Isso foi feito através da geração de dados
aleatórios (\textit{token} e senha), que são obtidos sempre que a Raspberry
é ligada, por estarem no \textit{bash\_rc}. Assim, todas as adições de sinais
feitas por uma Raspberry já serão relacionados com o respectivo paciente,
que poderá ter seus dados visualizados por parentes cadastrados.

A comunicação entre a Raspberry e o conversor é feita através do pacote em
Python \textbf{Adafruit\_ADS}, capaz de ler de até quatro canais ao mesmo tempo.
Contudo, uma ressalva: os valores enviados pelo sensor de temperatura não são
recebidos de uma maneira apresentável, por não estarem normalizados. Utilizamos
então a equação de Steinhart-hart:

$ 1/t = A + B * ln(R) + C[ln(R)]^3 $

Onde A, B e C são coeficientes, R é a resistência, e T a temperatura que
desejamos apresentar. Utilizamos os seguintes valores para os coeficientes:

$ A = 0.001129148; B = 0.000234125 ; C =  0.0000000876741 $

Para o cálculo do \textit{ln} utilizamos a função \textit{log1p} do Python, e
ressaltamos que tivemos diversos problemas de precisão, pois o Python arredonda
os resultados das operações de maneira grosseira em diversas situações. Por
fim, ajustamos outros parâmetros (como os utilizados na conversão da
resistência) utilizando outros resultados como base, de maneira experimental.

Para diminuir o consumo de banda, criamos um mecanismo que aborta o envio de
sinais redundantes.

\subsection{Arquitetura}
