\section{Termo de Abertura do Projeto}
Este sub capítulo tem como objetivo a formalização do projeto Unidade Móvel de
Identificação de Saúde e Socorro (UMISS). As informações contidas nos tópicos 
a seguir foram produzidos a fim de mostrar um resumo dos objetivos, riscos, 
limites e recursos, bem como mostrar o estudo de viabilidade do projeto. 
Tendo em vista as limitações de prazo, orçamento e infraestrutura dos 
participantes do projeto.
\subsection{Descrição do Projeto}
Para ter uma visão clara do projeto, foi usado o mapeamento de atividade no 
modelo 5W2H, que está descrito a seguir:

\begin{itemize}
    \item \textbf{What:} Sistema integrado com subsistemas de movimentação,
    monitoramento e estrutura. Desenvolvendo uma cadeira com possibilidade
    de movimentação e captação dos sinais enviados pelo paciente. E monitoramento
    através de um aplicativo mobile e de uma plataforma web.
    \item \textbf{Why:} Movimentar um indivíduo com mobilidade reduzida
    e enviar respostas recolhidas
    dos sensores para aplicativos e plataformas web dos seus respectivos
    cuidadores.
    \item \textbf{Where:} Em qualquer ambiente que tenha acesso a uma rede wifi.
    \item \textbf{When:} Durante o primeiro semestre de 2017.
    \item \textbf{Who:} Alunos dos cursos de Engenharias do Campus UnB Gama,
    que cursam a disciplina Projeto Integrador 2.
    \item \textbf{How:} Utilizando a orientação dos professores da Universidade
    de Brasília.
    \item \textbf{How Much:} Cerca de R\$ 3.000,00, quanto a equipamento, por
    cadeira.
\end{itemize}

\subsection{Propósito e justificativa do Projeto}
O projeto tem como objetivo oferecer à familiares de pacientes com mobilidade 
reduzida uma cadeira de rodas motorizada,
com fácil dirigibilidade 
e que tenha monitoramento sobre os sinais vitais do paciente,
a fim de avaliar suas condições de saúde e julgar se seu estado pode ser considerado normal
Neste caso, os cuidadores, que poderiam ser parentes, amigos 
próximos e médicos ou enfermeiros contratados, seriam notificados
do seu estado através de um site e um aplicativo. Todas essas respostas
devem respeitar um tempo máximo para o seu recebimento, tal como será
definido a seguir nas especificações do produto.

O desenvolvimento do produto se justifica uma vez que as cadeiras motorizadas 
são muito caras no mercado atual, além de que não existe nenhuma que possua 
um sistema de monitoramento da saúde do paciente, dificultando, então, seu 
resgate caso alguma emergência se faça presente. O que facilita em termos de 
preocupação e tempo a vida dos cuidadores, uma vez que não será mais necessária
a presença contínua ao lado do paciente, já que a cadeira fará esse papel.

\subsection{Restrições do Projeto}
As restrições do projeto UMISS são as descritas as seguir:
\begin{itemize}
    \item Limite de integrantes de acordo com os percentuais de inscritos na disciplina no semestre do trabalho;
    \item Limitação de custos do projeto, calibrado pelo montante que o grupo estará disposta a colaborar;
    \item O projeto está restrito ao tempo da disciplina de Projeto Integrador
   2 (06/03/2017 - 29/06/2017).
    \item Estar de acordo com as exigências do cliente, composto pelos Docentes
    da Universidade de Brasília da Unidade Gama.
\end{itemize}

\subsection{Riscos do Projeto}
Os principais riscos presentes na execução do Projeto UMISS, bem como suas medidas preventivas, estão explicitadas a seguir:

\begin{itemize}
    \item Falta de experiência dos membros dos projetos nos deveres de cada 
    subárea

Plano de ação: Cada subárea irá mapear e corrigir possíveis lacunas de 
conhecimento para a realização das tarefas

    \item Membro da equipe trancar ou abandonar a disciplina.

Plano de ação: Distribuir as tarefas entre os integrantes remanescentes de 
forma que não sobrecarregue nenhum dos membros da equipe.

    \item Falta de horário comum entre os membros

Plano de ação: Combinar encontros online e presenciais por meio das mídias de 
comunicação em horário não comercial.

    \item Falta de experiência em gerência de projeto

        Plano de ação: Obter  \textit{feedback} dos membros sobre a atuação e gerar planos de 
ação para melhorar cada ponto levantado.

    \item Possível descompromisso de membros da equipe

        Plano de ação: Realizar reuniões com o membro oferecendo suporte nas dificuldades levantadas

    \item Não cumprimento do cronograma sugerido

        Plano de ação: Ajuste das datas de cronograma de forma a acelerar prazos atrasados

    \item Alteração de requisitos de projeto

        Plano de ação: Repensar o escopo do projeto afim de adaptá-lo aos novos requisitos

    \item Queima ou danos em componentes do projeto

        Plano de ação: Compra de novos componentes ou recondicionamento de itens danificados

    \item Atraso na entrega de pedidos para o projeto

        Plano de ação: Procurar fornecedores locais para os componentes ou contratar frete expresso para entrega.
\end{itemize}



\subsection{Custos do Projeto}
De acordo com o \ref{relatoriogestao} da UnB o custo médio por aluno 
anual é de R\$12.100,00.

Considerando que o curso de Engenharia, no qual os alunos das 
disciplinas estão matriculados, exige 240 créditos para um aluno se formar em 
5 anos(10 semestres) e que cada crédito corresponde à 15 
horas/aula. Sendo assim o custo de formar um aluno na UnB é de 5 
vezes o custo anual(R\$12.100,00),ou seja, R\$60.500,00.

Diante disso, é possível concluir que se multiplicarmos 240(créditos) por 
15(horas/aula) obtemos o valor total de horas para se formar um aluno da UnB,
sendo esse valor total de 3.600 horas. Se dividirmos o custo total de 
formação(R\$60.500,00) pelo total de horas(3.600) teremos o custo da hora 
de um aluno da UnB, sendo esse valor R\$16,80 (reais/hora).

O projeto ocorrerá em um período de 15 semanas durante um semestre com data 
final a apresentação do segundo ponto de controle 2 Release, projeto que no qual cada estudante irá 
dedicar 10 horas por semana a disciplina, ou seja, haverá um esforço de 150 
horas para a disciplina durante o semestre.

Deste modo cada aluno irá custar R\$2.520 ao projeto, visto que R\$2.520,00 é o 
resultado da multiplicação de 150 (horas) e de R\$16,80 (reais/hora).

Sendo que a equipe é composta por 13 integrantes(Equipe),estudantes de 
Engenharia de Software. Sendo assim se multiplicarmos os 10 integrantes pelo 
custo de cada integrante, teremos o custo total do projeto, que é de 
R\$32.760,00.

\subsection{Stakeholders}

\subsubsection{Cliente}
Hospitais e clinicas que lidam com pessoas de mobilidade reduzida. Usuários e financiadores do usuário.

\subsubsection{Equipe de Gerência}
Membros do projeto que tem a responsabilidade de planejamento, monitoramento 
e controle do projeto, garantindo a excelência e o sucesso do produto. Além 
disso tem a responsabilidade de tomar decisões fundamentais dentro do projeto, 
sendo eles:

\begin{table}[]
\centering
\caption{Equipe de Gerentes}
\label{equipe_gerentes}
\begin{tabular}{|l|l|}
\hline
Posição              & Indivíduo      \\ \hline
Gerente geral        & Afonso Delgado \\ \hline
Gerente de qualidade & Dylan Guedes   \\ \hline
Gerente de produto   & Rafael Amado   \\ \hline
\end{tabular}
\end{table}

\subsubsection{Equipe de Desenvolvimento}
Membros do projeto que tem a responsabilidade de construir o produto e a 
documentação necessária para a finalização do produto. Sendo eles: Mariana 
Andrade, Lunara Alves, César Antônio, Johnson Andrade, Felipe Costa, Lucas 
Matheus, Gustavo Cavalcante, Wilton da Silva, Tiago Ribeiro e Nivaldo Lopo.


\subsubsection{Docentes}
Professores da matéria Projeto Integrador 2 do primeiro semestre de 2017, que 
tem como responsabilidade avaliar o andamento e finalização do projeto e seu 
produto final. Sendo eles: Alex Reis, Sebastien Rondineau, Rhander Viana e 
Luiz Laranjeira.

\subsection{Produto do Projeto}
As entregas do produto serão feitas em três partes, divididas em pontos de controle

\textbf{Ponto de controle 01}: Definição do escopo e projeto, bem como o 
alinhamento. Concepção e detalhamento da solução.

\textbf{Ponto de controle 02}: Projeto e desenvolvimento de subsistemas.


\textbf{Ponto de controle 03}: Integração de subsistemas e finalização do 
produto.

