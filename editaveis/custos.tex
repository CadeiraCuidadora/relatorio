\chapter{Custos}

Este capítulo detalhará as estimativas, planejamento e efetivação dos custos
referentes ao projeto UMISS. Para um melhor entendimento, os custos foram divididos
nas seções custos esperados e custos desprezados, e serão detalhadas abaixo.

\section{Custos desprezados}

A fim de obtermos uma estimativa mais realista do valor financeiro que deve ser
despendindo para a elaboração do projeto, a equipe decidiu por desconsiderar o custo
de alguns recursos, que embora necessários, na visão da equipe serviriam apenas
para inflar o orçamento final do projeto, e o deixar tendencioso. Estes custos são: internet, ambiente para
trabalho, custo de deslocamento e o próprio custo do aluno.
Estes custos, serão levantados e registrados no documento, mas não entrarão no valor
final que será apresentando como custo real do projeto.

\section{Custos esperados}

Custos esperados são aqueles que de fato serão comprados pela equipe e que estão detalhados mais abaixo
na tabela~\ref{tab:aquisicao}. A partir disto, foi definido pelos 13 integrantes do grupo que cada
um contribuirá com o valor de $R\$150,00$ inicialmente, possibilitando assim um orçamento
inicial de $R\$ 1.950,00$. Um integrante ficou responsável pela coleta do dinheiro.
Se identificada a necessidade de custos adicionais, será comunicado a todos e acordado
um novo valor de contribuição. Se ao fim do projeto restar alguma quantia, ela
será dividida igualmente entre todos.

\section{Plano de Aquisições e Controle de Custos}

O PMBOK traz o Plano de Gerenciamento das Aquisições como um componente do plano
de gerenciamento do projeto responsável por descrever como a equipe do projeto
fará a aquisição de produtos e serviços externos à organização.

\textbf{Identificação da necessidade de Aquisição:} Uma vez identificada a necessidade
de aquisição de um determinado bem ou serviço para o projeto, o membro que fez esta
identificação deve relatar aos demais integrantes a necessidade desta aquisição e
justificar a escolha da mesma.

\textbf{Levantamento de contra propostas:} Após receber a solicitação da aquisição
necessária para o projeto, os demais membros deverão procurar por fornecedores alternativos
que atendam a necessidade levantada na etapa anterior.

\textbf{Condução da aquisição:} Após ser discutido entre todos os membros do projeto,
  caso a necessidade de aquisição do recurso não seja viável, descarta-se a possibilidade
  de aquisição da mesma. Caso contrário será definido um responsável e uma data para realizar a aquisição.

\textbf{Controle e acompanhamento da aquisição:} Etapa na qual o responsável pela
aquisição fará o acompanhamento e monitoramento de eventuais mudanças no decorre
do processo de aquisição, conforme sejam necessárias.

\textbf{Registro de Aquisições:} Etapa de finalização da aquisição dos recursos,
garantindo que todo o processo de identificação, levantamento de contra propostas,
condução e controle foram efetuados de maneira eficaz e o registro de
aquisições na tabela abaixo:


\renewcommand\STprintnum[1]{\numprint{#1}}
\nprounddigits{2}
\begin{table}[h]
\begin{spreadtab}{{tabular}{clccc}}
\hline
@ID   & \multicolumn{1}{c}{@Item}                          & @Valor   & @Quant.  & @Sub-Total \\ \hline
01    & @Motor CC Bosch F 006 KM0 611                      & 215.00   & 2        & c2 * d2    \\
02    & @Raspberry Pi 3 Model B                            & 299.00   & 1        & c3 * d3    \\
03    & @Conversor A/D ADS115 16 bits                      & 042.80   & 1        & c4 * d4    \\
04    & @Botão Arcade Azul                                 & 009.00   & 1        & c5 * d5    \\
05    & @Amplificador INA 118/128                          & 000.00   & 3        & c6 * d6    \\
06    & @Amplificador TL084                                & 000.00   & 4        & c7 * d7    \\
07    & @Baterias Chumbo-Ácido 12V 100Ah Som Automotivo    & 68.00    & 3        & c8 * d8    \\
08    & @IRF3505 N-Channel MOSFET                          & 003.50   & 8        & c9 * d9    \\
09    & @Arduino UNO ATMega 328                            & 060.00   & 1        & c10 * d10    \\
10    & @Joystick de 3 Eixos                               & 013.50   & 1        & c11 * d11    \\
11    & @Rodas Aro 26''  para Bicicleta                    & 045.00   & 2        & c12 * d12    \\
12    & @Pneus Aro 26'' para Bicicleta                     & 027.00   & 2        & c13 * d13    \\
13    & @Cubos Roletados com Rolamento Selado              & 018.00   & 2        & c14 * d14    \\
14    & @Cadeira de Rodas                                  & 399.00   & 1        & c15 * d15    \\
15    & @Perfil de Aço 6m                                  & 020.00   & 1        & c16 * d16    \\
16    & @App para Playstore                                & 000.00   & 1        & c17 * d17    \\
17    & @Domínio para servidor                             & 000.00   & 1        & c18 * d18    \\ \hline
\multicolumn{4}{c}{@Total} & sum(e2:e18)\\ \hline
\end{spreadtab}
\caption{Aquisições do Projeto UMISS}
\label{tab:aquisicao}
\end{table}

