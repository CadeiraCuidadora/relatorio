\chapter{Riscos}
Seja qual for a decisão dentro de um projeto, desde o início com o
planejamento do projeto até a integração dos sistemas, são presentes os riscos
com níveis associados, uma vez que o acompanhamento deve ser feito de maneira contínua
durante todo o processo.
Os riscos possuem graus de probabilidade e impacto, tais que juntas classificam
cada risco, como elucidado na Tabela~\ref{tab:probimp} abaixo.

\begin{table}[h]
\centering
\vspace{0.5cm}
\begin{tabular}{|c|c|c|c|}
\hline
Probabilidade/Impacto & Leve            & Médio          & Grave \\
\hline
Alto                  & Risco elevado   & Risco extremo  & Risco extremo \\
Médio                 & Risco moderado  & Risco elevado  & Risco extremo \\
Baixo                 & Risco moderado  & Risco moderado & Risco elevado \\
\hline
\end{tabular}
\caption{Probabilidade/Impacto dos riscos}
\label{tab:probimp}
\end{table}

A Tabela~\ref{tab:probimp} é a chamada Matriz de Riscos, que auxilia na avaliação dos riscos
que são envolvidos na elaboração do projeto em questão. A Matriz de Riscos é
formada por dois eixos, o vertical sendo a probabilidade de ocorrer determinado risco,
e o horizontal sendo o eixo de impacto no projeto. De acordo com os principais tópicos
abordados pelo PMBOK (Project Management Book of Knowledge), é necessário que
haja o planejamento do gerenciamento no projeto, a identificação dos riscos
inerentes, realização de análise quantitativa (probabilidade) e análise qualitativa
(impacto), planejamento de feedback aos riscos e controle.

No que se refere aos riscos do projeto em questão, foram realizados
\textit{brainstormings} em reuniões da equipe, com o intuito de colocar em evidência os
possíveis riscos embutidos. De acordo com os dados coletados e análises realizadas,
foram formuladas as tabelas abaixo.

\begin{table}[h]
\centering
\vspace{0.5cm}
\begin{tabular}{|c|c|c|}
\hline
Riscos - Projeto Estrutural    & Leve            & Médio          \\
\hline
Acidentes causados por imperícia do usuário         & Médio & Leve \\
Falha estrutural                                    & Baixo & Grave \\
Falha na transmissão                                & Médio & Grave \\
Falha na mobilidade da cadeira                      & Médio & Médio \\
Falha de montagem                                   & Baixo & Médio \\
Falta de recurso para o cumprimento dos requisitos  & Médio & Médio \\
\hline
\end{tabular}
\caption{Probabilidade/Impacto dos riscos da Projeto Estrutural}
\label{tab:probimpest}
\end{table}

\begin{table}[h]
\centering
\vspace{0.5cm}
\begin{tabular}{|c|c|c|}
\hline
Riscos - Processamento de Sinais e Monitoramento       & Probabilidade & Impacto \\
\hline
Falha na transmissão de dados           & Baixo         & Grave \\
Instabilidade de sevidores e serviços   & Baixo         & Médio \\
Falha de aquisição de dados do usuário  & Médio         & Grave \\
\hline
\end{tabular}
\caption{Probabilidade/Impacto dos riscos de Processamento de Sinais e Monitoramento}
\label{tab:probimpcontrole}
\end{table}

\begin{table}[h]
\centering
\vspace{0.5cm}
\begin{tabular}{|c|c|c|}
\hline
Riscos - Controle e Alimentação        & Probabilidade & Impacto \\
\hline
Alta temperatura do motor   & Alto          & Grave \\
Vida útil da bateria        & Médio         & Médio \\
Descarregamento da bateria  & Baixo         & Médio \\
\hline
\end{tabular}
\caption{Probabilidade/Impacto dos riscos de Controle e Alimentação}
\label{tab:probimpalim}
\end{table}

De acordo com as análises obtidas conforme a Tabela~\ref{tab:probimp} de Matriz de Riscos,
cada risco foi agrupado em uma subárea do projeto, sendo possível identificar uma
média esperada de probabilidade e impacto de determinados riscos, sendo avaliado o
grau a classificação definida.

\begin{table}[h]
\centering
\vspace{0.5cm}
\begin{tabular}{|c|c|c|}
\hline
Riscos                    & Probabilidade & Impacto \\
\hline
Projeto Estrutural                 & Médio         & Médio \\
Processamento de Sinais e Monitoramento  & Médio         & Médio \\
Controle e Alimentação               & Médio         & Médio \\
\hline
\end{tabular}
\caption{Probabilidade/Impacto dos riscos de cada subárea}
\label{tab:probimparea}
\end{table}

Com os dados coletados, é necessária a elaboração do plano de ação com o
intuito de mitigar ou extinguir a ocorrência de algum risco analisado. Dessa forma, de
acordo com as tabelas a seguir, são desenvolvidas e listadas as propostas de soluções
para cada caso.

\begin{table}[h]
\centering
\vspace{0.5cm}
\begin{tabular}{|c|c|}
\hline
Riscos - Projeto Estrutural                                     & Solução \\
\hline
Acidentes causados por imperícia do usuário            & Fazer um manual do produto \\
Falha estrutural                                       & Fator de segurança adequado ao sistema \\
Falha na transmissão                                   & Validação teórica e experimental do sistema \\
Falha na mobilidade da cadeira                         & Validação do rolamento e boa lubrificação do sistema \\
Falha de montagem                                      & Metodologia de montagem bem definida \\
\hline
\end{tabular}
\caption{Soluções dos riscos da área de Projeto Estrutural}
\label{tab:riscosubareaest}
\end{table}

\begin{table}[h]
\centering
\vspace{0.5cm}
\begin{tabular}{|c|c|}
\hline
Riscos - Processamento de Sinais e Monitoramento         & Soluções \\
\hline
Falha na transmissão de dados             & Redundância de transmissão de dados \\ 
Instabilidade de servidores e serviços    & Retorno para o usuário do estado do servidor \\ 
Falha de aquisição de dados do usuário    & Redundância de transmissão de dados e reposicionamento dos sensores \\
\hline
\end{tabular}
\caption{Soluções dos riscos da área de Processamento de Sinais e Monitoramento}
\label{tab:riscosubareacont}
\end{table}

\begin{table}[h]
\centering
\vspace{0.5cm}
\begin{tabular}{|c|c|}
\hline
Riscos - Controle e Alimentação & Soluções \\
\hline
Alta temperatura do motor     & Desligamento do sistema até o reestabelecimento da temperatura adequada \\ 
Vida útil da bateria          & Coeficiente de profundidade de descarga da bateria \\ 
Descarregamento da bateria    & Cumprir o requisito de tempo de uso \\
Queima de componentes         & Dimensionamento correto dos componentes com margem de segurança \\
\hline
\end{tabular}
\caption{Soluções dos riscos da área de Controle e Alimentação}
\label{tab:riscosubareaalim}
\end{table}

As soluções foram elaboradas de modo a encontrar maneiras de contornar os riscos
evidenciados em cada subsistema durante e a realização do projeto.
