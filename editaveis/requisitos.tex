\chapter{Requisitos}

Os requisitos de um projeto são as descrições do que o sistema deve fazer, os
serviços que oferece e as restrições a seu funcionamento \cite{sommerville}.
Além daquilo que o produto tem de fazer, os requsitos também são responsáveis
por determinar a qualidade que ele precisa apresentar \cite{robertson}.
Baseado nisto, este capítulo tem por objetivo listar os requisitos presentes
no projeto.

\section{Requisitos Gerais}

\begin{enumerate}
\item Desenvolver a estrutura da cadeira.
\item O objetivo do projeto é atender a portadores de mobilidade reduzida, especificamente os paraplégicos.
\item O sistema precisa estar conectado à rede de internet para funcionar.
\end{enumerate}

\section{Subsistema - Controle e Monitoramento}

\begin{enumerate}[resume*]
\item O controle de movimentação da cadeira se dará por meio de um joystick.
\item O sistema fará o monitoramento dos seguintes sinais vitais:
\begin{enumerate}[resume*]
\item Sinal 1
\item Sinal 2
\item Sinal 3
\end{enumerate}
\item O sistema possibilitará ao usuário chamar alguém através de um botão.
\item O sistema será integrado a um aplicativo.
\item O sistema deverá possuir tolerância à falhas.
\end{enumerate}

\section{Alimentação}

\section{Estrutura}

\begin{enumerate}[resume*]
\item Estabelecer capacidade máxima de carga.
\item A estrutura deve atender aos princípios de ergonomia presente na NBR 9050.
\end{enumerate}
