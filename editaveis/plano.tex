\chapter{Plano de Integração}
\label{chapter:plano}

Neste capítulo apresentamos nosso planejamento para a integração dos diferentes
subsistemas, que deve ocorrer na terceira etapa do projeto. Adaptamos os pontos
levantados no guia do PMBOK, adequando o plano ao contexto da disciplina, e
focando na integração de subsistemas de um projeto de engenharia, e não da área
de gestão de pessoas.

A integração dos subsistemas ocorrerá no tempo das aulas, contudo, caso
necessário, em outros horários o grupo se reunirá de forma a conseguir a
integração completa. Ainda, ressaltamos que o subsistema de Processamento de
Sinais e Monitoramento não é fortemente acoplado aos outros subsistemas, de
modo que a integração entre ele e o restante do projeto será fácil e não
carecerá de um grande esforço. Os outros dois subsistemas, Projeto Estrutural
e Controle e Alimentação, por outro lado, são bem acoplados, e um maior cuidado
será tomado a respeito desses dois subsistemas.

\section{Integração do subsistema Processamento de Sinais e Monitoramento}

Como mencionado, o subsistema de Processamento de Sinais e Monitoramento não
deverá apresentar grandes problemas na integração, principalmente por não ser
acoplado aos outros subsistemas. Os únicos componentes físicos desse subsistema
são os componentes eletrônicos relacionados a aquisição de sinais (sensores,
amplificadores, filtros e conversores) e o sistema embarcado (Raspberry Pi).
Os outros componentes (servidor remoto, cliente \textit{frontend} e
\textit{mobile}) estão na nuvem, e não causam impacto na integração com os
outros componentes físicos.

É esperado que os componentes físicos desse subsistema sejam alocados em um
compartimento de fácil manutenção, para que seja facilmente manuseado durante
os diversos testes. Além disso, esse compartimento deve disponibilizar saídas
para os fios, que serão então conectados a outros subsistemas, ou
disponibilizados para serem utilizados pelo paciente. Assim, a integração deve
ocorrer da seguinte forma:

\begin{enumerate}
    \item Alocação dos componentes eletrônicos de modo seguro, mas que ocupe
        o menor espaço possível, pois os outros subsistemas carecem de bastante
        espaço;
    \item Acoplamento do compartimento na parte inferior da cadeira (nos braços),
        parafusando-o (incluindo o \textit{case} da Raspberry Pi);
    \item Extensão e disponibilização dos cabos e dos sensores, para que sejam
        facilmente utilizados pelos pacientes;
    \item Extensão da bateria conectada a Raspberry Pi, e conexão entre ela e a
        bateria do subsistema de Controle e Alimentação.
\end{enumerate}

\section{Cronograma}

\begin{table}[!htbp]
    \centering
    \caption{Cronograma para a integração dos subsistemas.}
    \label{tab:cronogramaintegracao}
    \resizebox{\textwidth}{!}{%
        \begin{tabular}{|l|l|l|}
            \hline
            \textbf{Atividade} & \textbf{Responsável} & \textbf{Deadline} \\
            \hline
            Criar compartimento com sensores e embarcado & Afonso, Dylan, Gustavo, Tiago e Wilton & 08/06\\
            \hline
            Acoplamento do compoartimento na parte inferior da cadeira & Dylan e Afonso & 08/06 \\
            \hline
            Extensão e disponibilização dos cabos & Gustavo, Tiago e Wilton & 08/06 \\
            \hline
            Extensão da bateria da Raspberry Pi & Dylan e Afonso & 08/06 \\
            \hline
            Teste do subsistema de Processamento e Monitoramento & Afonso, Dylan, Gustavo, Tiago e Wilton & 08/06 \\

            \hline

        \end{tabular}
    }
\end{table}
