\begin{resumo}
 Pacientes com atividade motora reduzida, em certo grau, necessitam de observação constante a fim de evitar acidentes ou o surgimento de outros problemas. Além disso, alguns graus de paraplegia ou tetraplegia necessitam ,muitas vezes, da presença de um cuidador para movimentação da cadeira de rodas e captura de sinais vitais a fim de monitorar suas atividades. Tendo isso em foco, o trabalho visa o projeto de uma cadeira de rodas elétrica com sensores para captura de sinais biológicos do paciente e um sistema de monitoramento remoto a fim de facilitar o controle da cadeira e alertar cuidadores ou familiares de possíveis riscos à integridade física do paciente.

 \vspace{\onelineskip}
    
 \noindent
 \textbf{Palavras-chaves}: cadeira de rodas. elétrica. tetraplegia. monitoramento. sensores.
\end{resumo}
