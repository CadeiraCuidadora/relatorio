\chapter{Trabalhos Futuros}
\label{chapter:futuro}

O resultado final do projeto, como um todo, está presente na Figura
\ref{fig:cadeira_final}. De maneira geral, os subsistemas se integraram sem
maiores problemas, principalmente pela parte mais complexa da integração
ter sido feita no Ponto de Controle 2\footnote{Consideramos a integração entre
os sensores e os servidores a integração mais complexa do sistema.}. Contudo,
por tratar-se de um projeto protótipo, identificamos pontos de melhoria para
trabalhos futuros, que podem auxiliar interessados na continuação do projeto.
Esses pontos de melhoria são:
\begin{itemize}
    \item \textbf{Sistema de hard-realtime:} Acreditamos que a utilização de um
        SO que atenda requisitos de hard-realtime agregue valor considerável a
        solução. A utilização de um sistema como o Xenomai seria indispensável
        em um cenário real, onde cada segundo é precioso.
    \item \textbf{Utilização de compressão dos sinais:} Embora tenhamos
        desenvolvidos algoritmos que filtrem a emissão de dados redundantes,
        a compressão dos sinais antes do envio (e decompressão no destino)
        otimizariam ainda mais o uso da banda.
    \item \textbf{Módulo GPRS:} O sistema atual pode contar com a reconfiguração
        do Linux para ativar redes já conhecidas e disponíveis, contudo, em um
        cenário real, a utilização de redundância na rede seria indispensável.
    \item \textbf{Sensor de pressão sanguínea:} O sensor de pressão sanguínea
        traria mais informações de valor para o cliente final.
\end{itemize}
