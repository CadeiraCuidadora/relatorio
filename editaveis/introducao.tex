\chapter{Introdução}

Neste relatório apresentamos o andamento e finalização do projeto WheelShare
(antes chamado UMISS) que ocorreu entre os pontos de controle 1, 2 e 3.
Focaremos principalmente em questões práticas, e levantaremos os resultados
obtidos e os esperados.

O objetivo desta etapa é integrar os subsistemas desenvolvidos, com base no
Plano de Integração desenvolvido na segunda parte da disciplina. Desta
maneira, este relatório trata sobre os resultados obtidos ao longo das últimas
semanas de implementação e integração do Ponto de Controle 3.

Organizamos o relatório da seguinte forma: cada capítulo que segue será relativo
aos resultados da integração de um subsistema do projeto. No Capítulo
\ref{chapter:processamento} apresentamos o andamento e integração do subsistema
de Processamento de Sinais e Monitoramento, que contempla o \textit{middleware},
a aquisição de sinais, o servidor remoto (\textit{backend}), o cliente \textit{web}
(\textit{frontend}) e o aplicativo Android. No Capítulo \ref{chapter:estruturas}
será apresentado o andamento e integração do subsistema de Estruturas. No
Capítulo \ref{chapter:movimentacao} será apresentado o andamento e integração do
subsistema de Controle e Alimentação, responsável pela movimentação e
alimentação dos sistemas integrantes do projeto. No Capítulo
\ref{chapter:atividades} apresentamos as atividades feitas por cada membro do
grupo, e por fim, no Capítulo \ref{chapter:consideracoes} apresentamos nossas
considerações finais sobre o projeto e a disciplina.

