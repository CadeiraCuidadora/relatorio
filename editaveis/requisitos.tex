\chapter{Requisitos}

Os requisitos de um projeto são as descrições do que o sistema deve fazer, os
serviços oferecidos e as restrições a seu funcionamento \cite{sommerville}.
Além de descrever as necessidades a serem cumpridas, os requisitos também são
responsáveis por determinar a qualidade que deve ser apresentada \cite{robertson}.
Baseado nisto, este capítulo tem por objetivo listar os requisitos presentes
no projeto.

\section{Requisitos Gerais}

\begin{enumerate}
  \item Desenvolver a estrutura da cadeira.
  \item Atender portadores de mobilidade reduzida, especificamente os paraplégicos.
  \item O sistema precisa estar conectado à internet.
\end{enumerate}

\section{Subsistema - Controle e Monitoramento}

\begin{enumerate}[resume*]
  \item O controle de movimentação da cadeira se dará por meio de um \textit{joystick}.
  \item O sistema fará o monitoramento dos seguintes sinais vitais:
    \begin{enumerate}[resume*]
      \item Temperatura
      \item Frequência cardíaca
      \item Resistência galvânica da pele
    \end{enumerate}
  \item Os meios de captura dos sinais do paciente devem ser não invasivos.
  \item O sistema deverá ser capaz de tratar sinais extraídos e realizar as
    conversões necessárias para processamento.
  \item O sistema deve atualizar os dados no servidor com variações de 5\% do
    último valor recebido.
  \item Possibilidade de notificar algum parente.
  \item Interação com recursos via aplicativo.
  \item O sistema deverá ser capaz de apresentar o histórico de dados
    capturados ao usuário.
  \item O sistema deve ser capaz de reiniciar o processador no caso de erros de
    captura de sinais, evitando o travamento completo do sistema.
  \item O sistema mobile deve ser capaz de notificar algum responsável quando
    um dos módulos essenciais para o funcionamento não estiver funcionando corretamente.
  \item O sistema deve apresentar um tempo de resposta máximo de 30 segundos
    até que o evento crítico seja identificado;
  \item O sistema web deve funcionar nos navegadores chrome e firefox.
  \item O aplicativo mobile deverá funcionar nas versões 4.4 do Android em diante.
  \item As ferramentas utilizadas deverão ter suporte para Linux.
\end{enumerate}

\section{Alimentação}

\section{Estrutura}

\begin{enumerate}[resume*]
  \item Estabelecer capacidade máxima de carga.
  \item A estrutura deve atender aos princípios de ergonomia presente na NBR 9050\footnote{\url{http://www.ufpb.br/cia/contents/manuais/abnt-nbr9050-edicao-2015.pdf}}.
\end{enumerate}
