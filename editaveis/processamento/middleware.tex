\section{Middleware}

O \textit{middleware} do subsistema, uma Raspberry, tinha como resultados
esperados uma aplicação que pudesse, ao rodar no embarcado, receber sinais e
enviá-los de maneira correta ao servidor.

Os resultados esperados foram atingidos, via a aplicação
Shoelace\footnote{\url{https://github.com/cadeiracuidadora/shoelace}}, que
desenvolvemos para servir de abstração entre a aquisição dos dados do conversor
A/D e o envio dos resultados para um servidor remoto.

Definimos que uma regra de negócio deveria ser que toda Raspberry tivesse
um \textit{token} e uma senha incluída, e esses dados são então utilizados nas
requisições para os servidores. Isso foi feito através da geração de dados
aleatórios (\textit{token} e senha), que são obtidos sempre que a Raspberry
é ligada, por estarem no \textit{bash\_rc}. Assim, todas as adições de sinais
feitas por uma Raspberry já serão relacionados com o respectivo paciente,
que poderá ter seus dados visualizados por parentes cadastrados.

A comunicação entre a Raspberry e o conversor é feita através do pacote em
Python \textbf{Adafruit\_ADS}, capaz de ler de até quatro canais ao mesmo tempo.
Contudo, uma ressalva: os valores enviados pelo sensor de temperatura não são
recebidos de uma maneira apresentável, por não estarem normalizados. Utilizamos
então a equação de Steinhart-hart, e contornarmos o arredondamento de valores
que o Python efetua.

Para diminuir o consumo de banda, criamos um mecanismo que aborta o envio de
sinais redundantes. Todo sensor deve implementar seu próprio limiar, que é
utilizado como filtro, que julga se a medição deve ou não ser enviada ao
servidor \textit{backend}. Caso a nova medição seja diferente da última
medição enviada para o servidor em uma porcentagem maior que o limiar
estabelecido para o sinal, a nova medição é enviada. Dessa forma, dados
redundantes não são enviados, o que alivia o processamento da Raspberry e do
servidor, e possibilita maior escala do ecossistema, que passa a poder receber
mais clientes (útil em cenários reais).

