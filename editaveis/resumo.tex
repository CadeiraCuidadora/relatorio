\begin{resumo}
%  Pacientes com atividade motora reduzida, em certo grau, necessitam de observação constante a fim de evitar acidentes ou o surgimento de outros problemas. Além disso, alguns graus de paraplegia ou tetraplegia necessitam ,muitas vezes, da presença de um cuidador para movimentação da cadeira de rodas e captura de sinais vitais a fim de monitorar suas atividades. Tendo isso em foco, o trabalho visa o projeto de uma cadeira de rodas elétrica com sensores para captura de sinais biológicos do paciente e um sistema de monitoramento remoto a fim de facilitar o controle da cadeira e alertar cuidadores ou familiares de possíveis riscos à integridade física do paciente.

Pacientes com capacidade motora reduzida, em certo grau, necessitam de
observação contínua a fim de evitar acidentes ou outros problemas. Além
disso, em alguns casos, a presença de um cuidador é necessária para ajudar na
movimentação da cadeira de rodas e na captura de sinais vitais.
Tecnologias nesse campo não evoluem rápido o suficiente, não resolvem estes
cenários ao mesmo tempo, e, mais ainda, são custozas.
Neste trabalho nós apresentamos a UMISS, uma cadeira elétrica que extrai sinais
vitais, notifica eventos críticos, e se move sem intervenção de terceiros.
Com a UMISS nós esperamos criar uma solução de baixo custo, que permita ao
paciente cuidar de si mesmo de maneira segura.

 \vspace{\onelineskip}
    
 \noindent
 \textbf{Palavras-chaves}: cadeira de rodas. acessível. monitoramento. sensores.
\end{resumo}
