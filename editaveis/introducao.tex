\chapter{Introdução}

Neste relatório apresentamos o andamento e a progressão do projeto UMISS que
ocorreu entre os pontos de controle 1 e 2. Focaremos em questões mais práticas,
e levantaremos os resultados obtidos e os esperados.

O objetivo desta etapa é desenvolver os subsistemas separadamente, levando em
consideração a segurança e redundância de cada um, bem como a sua qualidade final.
Estes são desenvolvidos em ambientes diferentes, porém, com o foco na integração de cada
um, que será a próxima fase deste desenvolvimento. Desta maneira, este relatório
trata sobre os resultados obtidos ao longo das semanas de implementação
do ponto de controle 2.

A organização se dará da seguinte forma: cada capítulo que segue será relativo
a um subsistema do projeto. No Capítulo \ref{chapter:processamento}
apresentamos o andamento do subsistema de Processamento de Sinais e
Monitoramento, que contempla o \textit{middleware}, a aquisição de sinais, o
servidor remoto (\textit{backend}), o cliente \textit{web} (\textit{frontend})
e o aplicativo Android. Por fim, no Capítulo \ref{chapter:plano} traremos o
nosso planejamento para a integração final dos diferentes subsistemas, e
as considerações finais sobre a segunda parte do projeto.

No Capítulo \ref{chapter:estruturas} será apresentado o andamento do subsistema
de Estruturas, contemplando as modelagens em ambiente CAD, análises modais do
sistema estrutural, projeto e fabricação de sistemas de redução por atrito e
de sistemas de adaptação para acoplamento de motores CC, baterias e demais sistemas.

No capítulo \ref{chapter:movimentacao} será apresentado o andamento do subsistema
de Controle e Alimentação, responsável pela movimentação e alimentação dos sistemas
integrantes do projeto UMISS. Este capítulo contempla os dimensionamentos dos
sistemas de alimentação e bateria, dimensionamento dos motores para a carga necessária,
projeto de \textit{drivers} e pontes H para controle dos motores e do carregador
de bateria, que acompanhará o projeto.

Todos os planos de integração para o ponto de controle 3 encontram-se no \ref{chapter:plano}
para apresentar como serão realizados os trabalhos para a integração dos subsistemas
aqui apresentados em um projeto final.
